\documentclass[12pt]{article}

\usepackage{geometry}
\usepackage[T1]{fontenc}
\usepackage[utf8x]{inputenc}
\usepackage{amssymb, amsmath, amsfonts}
\usepackage{ngerman}

\usepackage{float, graphicx}
\usepackage{multicol}

% Seitenlayout der Ränder
\textwidth=170mm
\textheight=250mm
\hoffset= -20mm       % may need change
\voffset= -30mm       % may need change

\begin{document}

\thispagestyle{empty}
\null\vspace{40mm}
\begin{center}
{
\Large  Atmospheric Trace Gases
\footnote{\noindent Versuch F18, ausgeführt am 11.12.2017, Betreuer Katja Bigge, kurze besondere Auswertung}
}\\[15mm]

C. Blattgerste und J. Ziegler

\vspace{25mm}

\parbox{0.9\textwidth}{

\small  This experiment gives insight into the spectroscopic remote sensoring techniques
, which are widely used to study the composition of Earth's atmosphere.
Therefor we make use of the Multi-Axis Differential Optical Absorption Spectroscopy (MAX-DOS).
This application uses the different absorption of trace gases depending on the
wavelength of scattered light and especially it is possible to gain information about
the gas vertical distribution.
}
\end{center}

\vfill
Als besondere Auswertung testiert: Datum, Unterschrift:
\vspace{20mm}

%% Rueckseite des Titelblatts leer. Bei einseitigem Druck entfernen
\newpage
\null\thispagestyle{empty}

\newpage     % Inhaltsverzeichnis, koennte man bei langer Version machen
\tableofcontents
\addtocontents{toc}{\vspace{\baselineskip}}
\newpage


\section{Einleitung}
  \begin{multicols}{2}
    test ob das ganze funktioniert
    ansonsten hab ich jetz tmal geguckt ob das mit minipage geht
  \end{multicols}

\section{theoretischer Hintergrund}

\section{Aufbau}

\section{Durchführung}
  \subsection{Messinstrumente kalibrieren}
      For the exact measurement needs
    \subsubsection*{Offset}

    \subsubsection*{dark current}
      \begin{tabular}{c|c}
        exposure time [s] & dark current \\ \hline
        $60$ & 168,27 \\
        50 & 139,09 \\
        40 & 111,91 \\
        30 & 83,30 \\
        20 & 55,62 \\
        10 & 27,95
      \end{tabular}
      \begin{figure}[h]
      	\centering
      	\includegraphics[width=70mm]{datei-name}
      	\caption{Übersicht über den Versuchsaufbau \label{fig:Versuchsaufbau}}
      \end{figure}
    \subsubsection*{noise}
      \begin{multicols}{2}
        The total noice is the sum of the instrument noice and noice given by
        statistical fluctuations. At first the instrument noice can be measured
        for different exposures.
        \begin{tabular} {c|c}
          exposures & noice $\sigma_I$ \\ \hline
          $10 000$ & $262.75$\\
          $8 000$ & $255,66$\\
          $6 000$ & $211,11$\\
          $4 000$ & $169,94$\\
          $2 000$ & $121,01$
        \end{tabular}
        Due to this measurement the total noice can be measured.

      \end{multicols}

\section{Auswertung}

\section{Diskussion}


\end{document}
