\documentclass[12pt]{article}

\usepackage{geometry}
\usepackage[T1]{fontenc}
\usepackage[utf8x]{inputenc}
\usepackage{amssymb, amsmath, amsfonts}
% \usepackage{ngerman}

\usepackage{float, graphicx}

% Seitenlayout der Ränder
\textwidth=170mm
\textheight=250mm
\hoffset= -20mm       % may need change
\voffset= -30mm       % may need change

\begin{document}

\thispagestyle{empty}
\null\vspace{40mm}
\begin{center}
{
\Large  Atmospheric Trace Gases
\footnote{\noindent Versuch F18, ausgeführt am 11.12.2017, Betreuer Katja Bigge, kurze besondere Auswertung}
}\\[15mm]

C. Blattgerste und J. Ziegler

\vspace{25mm}

\parbox{0.9\textwidth}{

\small  This experiment gives insight into the spectroscopic remote sensoring techniques
, which are widely used to study the composition of Earth's atmosphere.
Therefor we make use of the Multi-Axis Differential Optical Absorption Spectroscopy (MAX-DOS).
This application uses the different absorption of trace gases depending on the
wavelength of scattered light and especially it is possible to gain information about
the gas vertical distribution.
}
\end{center}

\vfill
Als besondere Auswertung testiert: Datum, Unterschrift:
\vspace{20mm}

%% Rueckseite des Titelblatts leer. Bei einseitigem Druck entfernen
\newpage
\null\thispagestyle{empty}

\newpage     % Inhaltsverzeichnis, koennte man bei langer Version machen
\tableofcontents
\addtocontents{toc}{\vspace{\baselineskip}}
\newpage


\section{introduction}

\section{theory}





\end{document}
